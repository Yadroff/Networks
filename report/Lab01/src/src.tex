\section{Ход выполнения работы}

\subsection{Настройка маршрутизатора}

\begin{lstlisting}[language=bash, caption=Настройка имени]
    system identity set name=R1
\end{lstlisting}

\begin{lstlisting}[language=bash, caption=Создание нового пользователя]
    user/add name=checker password=PfxtvXtrth! group=full
\end{lstlisting}

\begin{lstlisting}[language=bash, caption=Настройка $IP$]
    ip/address/add address=192.168.29.1/24 interface ether2 disabled=no
\end{lstlisting}

\begin{lstlisting}[language=bash, caption=Настройка $DNS$]
    ip/dns/set servers=9.9.9.9 allow-remote-requests=yes
\end{lstlisting}

\begin{lstlisting}[language=bash, caption=Настройка $DHPS$]
    ip/pool/add name=dhcp-pool1 ranges= 192.168.29.50-192.168.29.150
    ip/dhcp-server/add address-pool=dhcp-pool1 disabled=no interface=ether2
    ip/dhcp-server/network/add address=192.168.29.0/24 dns-server=192.168.29.1 gateway=192.168.29.1
\end{lstlisting}

\begin{lstlisting}[language=bash, caption=Настройка $NAT$]
    ip/firewall/nat/add action=masquerade chain=srcnat out-interface=ether1
\end{lstlisting}

\begin{lstlisting}[language=bash, caption=Отключение ненужных соединений]
    ip/service/disable telnet,ftp,www,www-ssl,api,api-ssl
\end{lstlisting}

\begin{lstlisting}[language=bash, caption=Настройка фаерволла]
    /ip firewall filter
    add action=accept chain=input connection-state=established,related in-interface=ether1
    add action=accept chain=input in-interface=ether1 protocol=icmp
    add action=drop chain=input in-interface=ether1    
\end{lstlisting}

\subsection{Настройка коммутатора}

\begin{lstlisting}[language=bash, caption=Изменение пароля админа]
    enable secret admin29   
\end{lstlisting}

\begin{lstlisting}[language=bash, caption=Настройка имени]
    hostname SW1
\end{lstlisting}

\begin{lstlisting}[language=bash, caption=Создание пользователей]
    username admin priv 15 pass 0 admin29
    username checker priv 15 pass 0 PfxtvXtrth!    
\end{lstlisting}

\begin{lstlisting}[language=bash, caption=Настройка $ip$]
    interface vlan 1 
    no shutdown 
    ip address 192.168.29.254 255.255.255.0
    exit      
\end{lstlisting}

\begin{lstlisting}[language=bash, caption=Настройка $DNS$]
    ip name-server 192.168.29.1
    ip route 0.0.0.0 0.0.0.0 192.168.15.1        
\end{lstlisting}

\begin{lstlisting}[language=bash, caption=Настройка $SSH$]
    ip domain name faq8.ru
    crypto key generate rsa modulus 2048
    ip ssh version 2
    service password-encryption
\end{lstlisting}

\begin{lstlisting}[language=bash, caption=Отключение ненужных подключений]
    line vty 0 4
    transport input ssh
    no ip http server
    no ip http secure-server    
\end{lstlisting}